\chapter{Embedding latin trades into groups}
\label{chap:lower-bound}

In this chapter, we define $\gdist(n)$ and present a proof for the lower bound in Conjecture \ref{conj:main}. We proceed as Drápal and Kepka in \cite{DrapalKepka89} and \cite{DrapalKepka85}. Since their work is a bit older, they use old terminology and their proof seems to be difficult to understand. We attempt to redo the proof using modern terminology and make it more accessible.

Moreover, we were able to improve the constant in the lower bound from $e \approx 2.718$ to $3 \log_3(e) \approx 2.731$, which is the best constant known so far. The key part in doing so is Lemma \ref{lem:technical}.

There are other proofs of the lower bound available, but they are not as general as the one of Drápal and Kepka. The proof of Cavenagh and Wanless \cite{CavenaghWanless09} considers only planar latin bitrades and abelian groups. (On the other hand, it exhibits an interesting connection between determinants and permanents of certain matrices, which we omit.) Another proof by Cavenagh \cite{Cavenagh03} restricts itself to cyclic groups only. %
%\footnote{The author suspects that the proof of Cavenagh is not complete -- he silently assumes that determinant of a certain matrix is nonzero, but the proof is missing.}
The proof of Drápal and Kepka works with all latin bitrades and all finite groups.

%%%
%%%
%%%
\section{Sparse matrices}

\begin{defn}
A matrix is \emph{sparse} if its elements are from $\{0,1\}$ and there is at most one occurrence of 1 in every row.
\end{defn}

We denote by $\overbar{M}_r^c$ the matrix obtained from $M$ by excluding row $r$ and column~$c$.

\begin{lem}
\label{lem:sparse2}
Let $M_1,M_2$ be sparse matrices with the same number of rows such that $M = (M_1,M_2)$ is a square block matrix. Then $\det(M) \in \{-1,0,1\}$.
\end{lem}
\begin{proof}
Let $c_1,c_2$ be the number of columns in $M_1$ and $M_2$ respectively. The proof is by induction on $n := c_1+c_2$. The case with $n=1$ is trivial. Therefore assume $n>1$.

There are at most two ones in every row of M.
\begin{cosyitemize}
	\item If there is a row with zeros only, then $\det(M) = 0$.
	\item If every row contains exactly two ones, then
		\begin{cosyeqnarray}
			v = (\underbrace{1,\dots,1}_{c_1},\underbrace{-1,\dots,-1}_{c_2}) \nonumber
		\end{cosyeqnarray}%
		is such that $Mv^T = 0$. Thus $M$ is singular and $\det(M) = 0$.
	\item Otherwise there is a row $r$ which contains only a single one in column $c$. Then expanding the determinant by row $r$ yields
	 \begin{cosyeqnarray}
	 	\det(M) = \pm \det(\overbar{M}_r^c).
	 \end{cosyeqnarray}%
	The matrix $\overbar{M}_r^c$ consists also of two sparse matrices, thus the result follows by induction.
\end{cosyitemize}
\end{proof}

\begin{lem}
\label{lem:sparse3}
Let $M_1,M_2,M_3$ be sparse matrices of sizes $n \times c_1, n \times c_2, n \times c_3$ such that $M = (M_1,M_2,M_3)$ is a square block matrix. Let $k_i$ denote the number of ones in column $i$. Then
\begin{cosyeqnarray}
	|\det(M)| \leq \prod_{i \in [c_1]} k_i.
\end{cosyeqnarray}%
\end{lem}
\begin{proof}
The proof is by induction on $c_1$. Denote $M = (m_{i,j})_{i,j \in[n]}$.
\begin{enumerate}
	\item If $c_1 = 0$, then $|\det(M)| \leq \prod_{i\in\emptyset} k_i = 1$ holds by Lemma \ref{lem:sparse2}.
	\item Otherwise expand by the first column:
	\begin{cosyeqnarray}
		|\det(M)| \leq \sum_{i\in[n]} m_{i,1}\,|\det(\overbar{M}_i^1)| \leq k_1 \prod_{i \in [2, c_1]} k_i.
	\end{cosyeqnarray}%
	The last inequality holds since there are $k_1$ nonzero summands and the product majorizes each subdeterminant from induction.
\end{enumerate}
\end{proof}

For our final result we need the following technical lemma:

\begin{lem}
\label{lem:technical}
Let $n$ be a positive integer and $k_1 + \dots + k_m = n$ for positive integers $m$ and $k_i$, $i \in [m]$. Then
\begin{cosyeqnarray}
	\prod_{i \in [m]} k_i \leq 3^{n/3}.\label{eq:prod-ki}
\end{cosyeqnarray}%
\end{lem}
\begin{proof}
For $n=1$ it holds trivially, let us assume $n > 1$. Let $m$ be the greatest such that the maximum in (\ref{eq:prod-ki}) is attained, and $(k_1, \dots, k_m)$ be the lexicographically smallest $m$-tuple for which this happens. Observe the following inequalities, in which both sides have the same sum:
\begin{cosyitemize}
	\item $2 \cdot (k-2) \geq k$ for $k \geq 4$, therefore $k_i \leq 3$;
	\item $(1+k) > 1 \cdot k$, therefore $k_i > 1$;
	\item $3 \cdot 3 > 2 \cdot 2 \cdot 2$, therefore there are at most two twos amongst $k_i$.
\end{cosyitemize}%
Thus there are three possibilities:
\cosyalig{
&n =3k &\Rightarrow&\ k_1 = \dots = k_m = 3 &\Rightarrow& \prod k_i = 3^{n/3} \\
&n =3k+2 &\Rightarrow&\ k_1 = 2,\ k_2 = \dots = k_m = 3 &\Rightarrow& \prod k_i = 2\cdot3^{(n-2)/3} \\
&n =3k+4 &\Rightarrow&\ k_1 = k_2 = 2,\ k_3 = \dots = k_m = 3 &\Rightarrow& \prod k_i = 4\cdot3^{(n-4)/3} 
}%
where the products are over $i \in [m]$. Each of them is less than or equal to $3^{n/3}$, thus we are done.
\end{proof}

\begin{lem}
\label{lem:3-block-sparse-det}
Let $M_1,M_2,M_3$ be sparse matrices such that $M = (M_1,M_2,M_3)$ is a  square block $n \times n$ matrix. Then
\begin{cosyeqnarray}
 	|\det(M)| \leq 3^{n/3}.
\end{cosyeqnarray}% 
\end{lem}
\begin{proof}
Let $c_1$ be the number of columns of $M_1$ and $k_i$ the number of ones in the column $i$. Since $M_1$ is sparse, surely $\sum_{i \in [c_1]} k_i \leq n$. The proof is finished by combining Lemmas \ref{lem:sparse3} and \ref{lem:technical}:
\begin{cosyeqnarray}
	|\det(M)| \leq \prod_{i \in [c_1]} k_i \leq 3^{n/3}.
\end{cosyeqnarray}%
\end{proof}

%%%
%%%
%%%
\section{The trade matrix}
Recall that
\begin{itemize}
	\item $R = \{r_1,\dots,r_{|R|}\}$ denotes the set of rows,
	\item $C = \{c_1,\dots,c_{|C|}\}$ denotes the set of columns, and
	\item $S = \{s_1,\dots,s_{|S|}\}$ denotes the set of symbols.
\end{itemize}
Without loss of generality assume that these sets are disjoint and let
\cosyalign{
X = R \cup C \cup S = \{r_1,\dots,r_{|R|}, c_1,\dots,c_{|C|}, s_1,\dots,s_{|S|}\}.
}%

\begin{defn}
Let $T$ be a latin trade. Fix ordering of elements of $X$ as above and choose an ordering on $T$.

We define a matrix $M = (m_{i,j})_{i \in T, j \in X}$ of size $|T| \times |X|$ such that
\begin{cosyeqnarray}
	t = (r,c,s) \in T \Rightarrow
	\begin{cases}
		m_{t,r} = m_{t,c} = m_{t,s} = 1, & \\
		m_{t,x} = 0 & \textrm{ for } x \in X \setminus \{r,c,s\}.
	\end{cases}
\end{cosyeqnarray}%
\end{defn}
We call it the \emph{trade matrix} and denote by $M_T$.

\begin{lem}
Suppose that all elements of $X$ are used in a connected latin trade $T$. Then
\cosyalign{
	\Ker(M_T) = \{(\underbrace{x, \dots, x}_{|R|}, \underbrace{y, \dots, y}_{|C|}, \underbrace{z, \dots, z}_{|S|}) \mid x+y+z = 0\} \label{eq:KerMT}
}%
where $M_T$ is considered as a matrix over $\Q$.
\end{lem}
\begin{proof}
The proof is divided into several steps.

%
%
\item \textbf{Step 1.}
Denote the set in (\ref{eq:KerMT}) by $V$, obviously $V \subset \Ker(M_T)$. Take a vector $v \in M_T$, from definition it is a solution of $M_Tv^T = 0$. Let us denote coordinates of $v$ by defining $f_v:X \rightarrow \Q$ such that
\begin{align}
	v = \big(\underbrace{f_v(r_1), \dots, f_v(r_{|R|})}_{|R|}, \underbrace{f_v(c_1), \dots, f_v(c_{|C|})}_{|C|}, \underbrace{f_v(s_1), \dots, f_v(s_{|S|})}_{|S|}\big).
\end{align}
Choose $(x_0, y_0, z_0) \in T$ such that $f_v(z_0)$ is maximal. By setting
\begin{align}
	w = v+\big(\underbrace{-f_v(x_0), \dots\ }_{|R|}, \underbrace{-f_v(y_0), \dots\ }_{|C|},\underbrace{f_v(x_0)+f_v(y_0), \dots\ }_{|S|}\big)
\end{align}
we obtain a solution in which $f_w(x_0) = f_w(y_0) = f_w(z_0) = 0$, and moreover $\max \{f_w(z) \mid z\in S\} = 0$. Now, for the other inclusion $\Ker(M_T) \subset V$ it suffices to prove that $w$ must be the zero vector.

Set $f=f_w$. Our goal is thus to prove $f(x) = 0$ for all $x \in X$. To shorten notation, we denote $f(x,y,z) = f((x,y,z)) = (f(x),f(y),f(z))$.

%
%
\item \textbf{Step 2.}
For all $(x,y,z) \in T$ holds $f(x)+f(y)+f(z)=0$. Since $0$ is the largest element of $\{f(z) \mid z \in S\}$ and $T'$ occupies the same cells as $T$, for all $(x,y,z) \in T \cup T'$ we have
\begin{align}
	f(x) + f(y) \geq 0.
\end{align}
In steps 3 and 4 we prove that if $t \in T\cup T'$ and $f(t) = (0,0,0)$, then
\begin{align}
	f(\sigma_Y(t)) = (0,0,0)
\end{align}
for $Y \in \{R, C, S\}$. Since the bitrade is connected, Lemma \ref{lem:connected-sigma} implies that $f(t) = (0,0,0)$ for all $t \in T \cup T'$. Because all symbols are used in the bitrade, from that we have the desired $f(x) = 0$ for all $x \in X$.

%
%
\item \textbf{Step 3.}
Let $(x,y,z) \in T'$ such that $f(x,y,z) = (0,0,0)$. Then
\begin{align}
	f(\sigma_R(x,y,z)) = f(x',y,z) = (f(x'), 0, 0)
\end{align}
for some $x' \in R$ such that $(x',y,z) \in T$. Therefore $f(x') + 0 + 0 = 0$. Similarly for $\sigma_C$ and $\sigma_S$.

%
%
\item \textbf{Step 4.}
Now let $(x_1,y_1,z) \in T$ such that $f(x_1,y_1,z)=(0,0,0)$. Consider a chain of elements in $T \cup T'$:
\begin{align*}
	(x_1,y_1,z) &\in T &\xrightarrow{\sigma_C}& &(x_1,y_2,z) &\in T' & \xrightarrow{\sigma_R} \\
	(x_2,y_2,z) &\in T &\xrightarrow{\sigma_C}& &(x_2,y_3,z) &\in T' & \xrightarrow{\sigma_R} \\
	(x_3,y_3,z) &\in T &\xrightarrow{\sigma_C}& &(x_3,y_4,z) &\in T' & \dots
\end{align*}
From $f(x) + f(y) \geq 0$ we have
\begin{align*}
	f(x_1) + f(y_1) &= 0, \quad f(x_1) + f(y_2) \geq 0, \\
	f(x_2) + f(y_2) &= 0, \quad f(x_2) + f(y_3) \geq 0, \\
	f(x_3) + f(y_3) &= 0, \quad f(x_3) + f(y_4) \geq 0,\ \dots, 
\end{align*}
which yields
\begin{align*}
	0 &= f(x_1) \geq f(x_2) \geq f(x_3) \geq \dots \\
	0 &= f(y_1) \leq f(y_2) \leq f(y_3) \leq \dots
\end{align*}
According to Lemma \ref{lem:sigma-cycles}, the chain is a cycle, and thus all terms in the inequalities equal to zero. Especially $f(\sigma_C(x_1,y_1,z)) = (0,0,0)$.

The result for $\sigma_R$ can be obtained by changing the roles of $\sigma_R$ and $\sigma_C$.

For $\sigma_S$, consider a cycle generated by alternating $\sigma_C$ and $\sigma_S$. We already know that $f(\sigma_S\sigma_C(x_1,y_1,z)) = \sigma_S(0,0,0) = (0,0,0)$. Therefore all elements in the cycle are mapped to $(0,0,0)$. By reversing the cycle we get the result.

\end{proof}

\begin{cor}
\label{cor:rank-mt}
Suppose that all elements of $X$ are used in a connected latin trade $T$. Then the trade matrix $M_T$ has rank $|X|-2$ over $\Q$.
\end{cor}

\begin{note}
Suppose that $M$ is a square submatrix of $M_T$ of rank $|X|-2$. It must have been obtained from $M$ by deleting two columns and some rows. These two columns cannot be both from $R$. If they were, then
\cosyalign{
	(\underbrace{0, \dots, 0}_{|R|-2}, \underbrace{y, \dots, y}_{|C|}, \underbrace{-y, \dots, -y}_{|S|})
}%
would be solutions of $Mv^T = 0$, which contradicts the regularity of $M$. Therefore the deleted columns must be from two different sets from $R$, $C$, $S$.
\end{note}


\begin{cor}
\label{cor:size-of-t}
$|T| \geq |X|-2$.
\end{cor}

%%%
%%%
%%%
\section{Homotopies from latin trades to groups}

\begin{defn}
\emph{A Cayley table} of a group $G$ is a latin square $L \subset G \times G \times G$ such that $(r,c,s) \in L$ if and only if $r \cdot c = s$. We denote it by $G$ when no confusion can arise.
\end{defn}

For abelian groups, we will use different definition:

\begin{defn}
\emph{A Cayley table} of an abelian group $(G,+)$ is a latin square $L \subset G \times G \times G$ such that $(r,c,s) \in L$ if and only if $r + c + s = 0$.
\end{defn}

It easily follows that for abelian groups the Cayley tables by first and second definition are isotopic. Because we will be interested in existence of homotopies into Cayley tables, it does not matter which definition we use.

\begin{defn}
Let $z(G)$ denote the size of the smallest trade $T$ such that there exist a non-trivial homotopy from $T$ to $G$. Let \emph{$z(n)$} be the minimum across all groups of order $n$.
\end{defn}

\begin{lem}
\label{lem:p-leq-det}
Let $p$ be a prime, $T$ connected latin trade using all symbols in $X$, $h: T \rightarrow \Z_p$ non-trivial homotopy and $M$  square submatrix of $M_T$ of rank $|X-2|$. Then $p \leq \det(M)$.
\end{lem}
\begin{proof}
Let $h = (h_R,h_C,h_S)$. Then
\begin{align*}
	v = \big(\underbrace{h_R(r_1), \dots, h_R(r_{|R|})}_{|R|},\underbrace{h_C(c_1), \dots, h_C(c_{|C|})}_{|C|}, \underbrace{h_S(s_1), \dots, h_S(s_{|S|})}_{|S|}\big)
\end{align*}
is a solution of $M_Tv^T = 0$ over $\Z_p$. Without loss of generality assume that columns $r_1$, $c_1$ were deleted from $M_T$ to obtain $M$ (see the note after Corollary \ref{cor:rank-mt}). Also suppose that $h_R(r_1) = 0 = h_C(c_1)$, otherwise we can set $h := (h_R', h_C', h_S')$ with
\cosyalig{
	h_R' = h_R - h_R(r_1), \quad h_C' = h_C - h_C(c_1), \quad h_S' = h_S + h_R(r_1) + h_C(c_1).
}%
Then
\begin{align*}
	w = \big(\underbrace{h_R(r_2), \dots, h_R(r_{|R|})}_{|R|-1},\underbrace{h_C(c_2), \dots, h_C(c_{|C|})}_{|C|-1}, \underbrace{h_S(s_1), \dots, h_S(s_{|S|})}_{|S|}\big)
\end{align*}
is a solution of $Mw^T = 0$ over $\Z_p$ which is non-trivial, since $h$ is non-trivial.

Therefore $\det(M) = 0$ in $\Z_p$, which means $p \mid \det(M)$. Because $M$ is reqular, $\det(M) \ne 0$ and we are done.
\end{proof}

\begin{lem}
\label{lem:lower-bound-zp}
$3 \log_3(p) \leq z(p)$.
\end{lem}%
\begin{proof}
Let $T$ be a latin trade such that $|T| = z(p)$ and there exists a non-trivial homotopy $T \rightarrow \Z_p$. From Lemma \ref{lem:p-leq-det} we know that $p \leq \det(M)$ for a submatrix of $M_T$ of rank $|X|-2$.

We can write $M_T = (M_1, M_2, M_3)$ for sparse matrices $M_1$, $M_2$, $M_3$, and therefore any submatrix $M$ of $M_T$ is of the same type. Thus from Lemma \ref{lem:3-block-sparse-det} and Corollary \ref{cor:size-of-t}
\cosyalign{
	p \leq \det(M) \leq 3^{(|X|-2)/3} \leq 3^{z(p)/3}.
}%
\end{proof}

\begin{lem}
\label{lem:nontrivial-homotopy-normal-subgroup}
Let $H$ be a normal subgroup of $G$ and $h: T \rightarrow G$ is a non-trivial homotopy. Then there exists a non-trivial homotopy $h_1: T \rightarrow H$ or $h_2: T \rightarrow G/H$.
\end{lem}
\begin{proof}
Let $\pi: G \times G \times G \rightarrow G/H \times G/H \times G/H$ be the natural projection. If $h_2 := \pi h$ is non-trivial, we are done. Otherwise $\Img(h) \subset H \times H \times H$ and we can set $h_1 := h$.
\end{proof}

\begin{lem}
\label{lem:zG-is-zp}
Let $G$ be a group of odd order. Then
\cosyalign{
	z(G) = \min \{z(p) \mid \mbox{ prime $p$ divides $|G|$} \}.
}%
\end{lem}
\begin{proof}
Lemma \ref{lem:nontrivial-homotopy-normal-subgroup} together with odd order theorem imply the "$\geq$" inequality. The other one follows from Cauchy's theorem.
\end{proof}

%%%
%%%
%%%
\section{Lower bound for gdist(n)}

\begin{defn}
A latin trade $T$ \emph{can be embedded} (or \emph{is embeddable}) in a group $G$ if there exists an injective homotopy from $T$ to $G$.

Let \emph{$\gdist(G)$} denote the size of the smallest trade embeddable in $G$ and let \emph{$\gdist(n)$} be the minimum across all groups of order $n$.
\end{defn}

From the tabular point of view, a latin trade $T$ can be embedded in a group $G$ if we can find an isotopic copy of the partial latin square $T$ inside of the Cayley table of $G$. The next lemma states that $\gdist(G)$ is the smallest number of cells in the Cayley table of $G$ that have to be changed in order to get another latin square. Therefore $\gdist(n)$ is the minimal "Hamming distance" between groups of order $n$ and latin squares.

\begin{lem}
Let $G$ be a group. Then
\cosyalign{
	\gdist(G) = \min \{|G \setminus L| : L \subset G \times G \times G \mbox{ is a latin square}, L \ne G \}.
}%
\end{lem}
\begin{proof}
$(G \setminus L, L \setminus G)$ is a latin bitrade, hence "$\leq$" holds. On the other hand, if $(T,T')$ is a latin bitrade embeddable in $G$ via injective homotopy $h$, then $G \setminus h(T) \cup h(T')$ is a latin square.
\end{proof}

\begin{exmp}
See Figure \ref{fig:z5-embedded-trade}.
\end{exmp}%

\begin{figure}[htb]
	\centering
	\begin{minipage}{.35\linewidth}
		\begin{center}
		\begin{tabular}{| c c c c |}
			\hline
				0 &   & 3 & 4 \\
				  & 3 & 4 & 0 \\
				3 & 0 &   &   \\
			\hline
		\end{tabular} \\
		\bigskip
		$T$
		\end{center}
	\end{minipage}
	\begin{minipage}{.35\linewidth}
		\begin{center}
		\begin{tabular}{| c c c c |}
			\hline
				3 &   & 4 & 0 \\
				  & 0 & 3 & 4 \\
				0 & 3 &   &   \\
			\hline
		\end{tabular}\\
		\bigskip
		$T'$
		\end{center}
	\end{minipage}\\
	\bigskip%
	\begin{minipage}{.35\linewidth}
		\begin{center}
		\begin{tabular}{| c c c c c |}
			\hline
				\M0 & 1 & 2   & \M3 & \M4 \\
				1   & 2 & \M3 & \M4 & \M0 \\
				2   & 3 & 4   & 0   & 1   \\
				\M3 & 4 & \M0 & 1   & 2   \\
				4   & 0 & 1   & 2   & 3   \\
			\hline
		\end{tabular}\\
		\bigskip
		$\Z_5$
		\end{center}
	\end{minipage}
	\begin{minipage}{.35\linewidth}
		\begin{center}
		\begin{tabular}{| c c c c c |}
			\hline
				\M3 & 1 & 2   & \M4 & \M0 \\
				1   & 2 & \M0 & \M3 & \M4 \\
				2   & 3 & 4   & 0   & 1   \\
				\M0 & 4 & \M3 & 1   & 2   \\
				4   & 0 & 1   & 2   & 3   \\
			\hline
		\end{tabular}\\
		\bigskip
		$L$
		\end{center}
	\end{minipage}
	\caption{Latin trade $T$ embedded in $\Z_5$ and the corresponding latin square $L$.}
	\label{fig:z5-embedded-trade}
\end{figure}

\begin{thm}
\label{thm:lower-bound}
Let $p$ be the smallest prime factor of $n \geq 2$. Then
\cosyalign{
	3\log_3(p) \leq \gdist(n).
}%
\end{thm}%
\begin{proof} \side{$\gdist(2k) = 4$}%
If $n$ is even, then $p=2$, $\gdist(n) = 4$ and the inequality holds. Otherwise by Lemmas \ref{lem:lower-bound-zp} and \ref{lem:zG-is-zp} there is a prime factor $p_0$ of $|G|$ such that
\begin{align}
	3 \log_3(p_0) \leq z(p_0) = z(G) \leq \gdist(G),
\end{align}
where the last inequality is trivial from definition. The fact that $p \leq p_0$ finishes the proof.
\end{proof}
