\chapter*{Conclusion}
\addcontentsline{toc}{chapter}{Conclusion}

While this thesis has answered some of open problems that are concerned with spherical latin bitrades, not all such problems have been solved. Some of them are mentioned in Chapter \ref{chap:bounds}, but we didn't formulate the most famous one -- the Barnette's conjecture. In language of latin bitrades, it can be stated as follows:

\begin{conj*}[Barnette]
The graph of a spherical connected latin bitrade has a Hamiltonian cycle.
\end{conj*}%

\noindent
It has been computationally proved that if a counterexample exists, it must have at least 86 vertices. For more about the conjecture we refer the reader to \cite{Hertel05}.

\bigskip

As for the values of $\gdist(n)$, let us reiterate three conjectures, which would lead to their full classification:

\begin{conj*}
$\gdist(n) = \min \{\gdist(p) \mid \mbox{prime } p \mbox{ divides } n\}$.
\end{conj*}%

\begin{conj*}
$\gdist(p) = t(p)$.
\end{conj*}%

\begin{conj*}
$\spb(n)-1 \leq \hat t(n) \leq \spb(n)$.
\end{conj*}%

The last one appears to be the most challenging. A solution to it would require to involve Padovan sequence to capture the behavior of $\spb(n)$, and simultaneously to work with trade matrices and its determinants, as it is the only way to describe $\hat t(n)$ explicitly (at least for now). Maybe a solution could make use of permanent of a trade matrix explained in \cite{CavenaghWanless09} -- the permanent can be interpreted as number of perfect matchings of certain objects on the graph of latin bitrade. Such a combinatorial interpretation could provide a link to combinatorial properties of Padovan sequence. However, at this point, these are only hypothetical considerations.